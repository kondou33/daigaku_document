\documentclass[uplatex]{jsarticle}
\usepackage{amsmath}
\usepackage[dvipdfmx]{graphicx}

\setcounter{tocdepth}{3}
\usepackage{float}
\usepackage{moreverb}
\usepackage{lscape}
%\pagestyle{empty}
%\usepackage{wrapfig}
%\usepackage{url}
%\usepackage{EasyLayout}

\usepackage{ascmac}
%\usepackage{fancybx}

%\pagestyle{myheadings}



\begin{document}


\title{第6回目の課題}
\author{25G1051近藤巧望}
%\date{2015年11月13日}
\maketitle


\section{演習問題: 行内に数式を記述する場合}
% この下の行に入力
オームの法則は,電圧$V$[V] と電流$I$[A],抵抗$R$ [$\Omega$] の間に成立しる関係: $V=IR$で表す
ことができる.
\section{演習問題: 独立した行に数式を記述する場合}
% この下の行に入力
オームの法則は,電圧$V$[V] と電流$I$[A],抵抗$R$ [$\Omega$] の間に成立しる関係式で
(\ref{eq:VIR}) 式で表すことができる.\\
\begin{equation}
V=IR~~~\label{eq:VIR}
\end{equation}
ここで,本実験では,$R=1$ [$\Omega$] の抵抗を用いる.

\section{表の記述}

\begin{table}[htb]
\centering
\caption{抵抗 $R=1.0$ [$\Omega$] の場合の電流値$I$ [A] と電圧値$E$ [V].}
\label{table:EVR1}
\begin{tabular}{cc}
\hline
電流 $I$ [A] & 電圧 $E$ [V]\\\hline\hline
$1.0$ & $1.0$ \\\hline
$2.0$ & $2.0$ \\\hline
$3.0$ & $3.0$ \\\hline
$4.0$ & $4.0$ \\\hline
$5.0$ & $5.0$ \\\hline
$6.0$ & $6.0$ \\\hline
\end{tabular}
\end{table}



\end{document}





