\documentclass[uplatex]{jsarticle}
\usepackage{amsmath}
\usepackage[dvipdfmx]{graphicx}

\setcounter{tocdepth}{3}
\usepackage{float}
\usepackage{moreverb}
\usepackage{lscape}
%\pagestyle{empty}
%\usepackage{wrapfig}
%\usepackage{url}
%\usepackage{EasyLayout}

\usepackage{ascmac}
%\usepackage{fancybx}

%\pagestyle{myheadings}


\begin{document}

\title{情報工学概論第8回課題発展課題8}
\author{25G1051近藤巧望}
%\date{2025年5月22日}
\maketitle

\section{選択ソート}
\subsection{時間計算量}

最良、平均、最悪すべてのケースで時間計算量は O(n²) と一定であり、特にデータが整列されていても効率は変わらない.
\subsection{空間計算量}

空間計算量は O(1) で、追加のメモリをほとんど使わないインプレースなアルゴリズムである.

\section{クイックソート}
\subsection{時間計算量}

平均および最良のケースでは時間計算量が O(n log n) と非常に高速だが、
最悪の場合(すでに整列されたデータなど)には O(n²) になる可能性がある.

\subsection{空間計算量}

空間計算量は O(log n) で、再帰呼び出しによりスタックメモリを使用する.

\section{マージソート}
\subsection{時間計算量}

時間計算量は、分割とマージの両方に O(n log n) の計算量がかかるため、最良・平均・最悪すべて O(n log n) である.

\subsection{空間計算量}

マージの過程で一時的な配列が必要になるため、空間計算量は O(n) である.

\section{バブルソート}
\subsection{時間計算量}

最良の場合(すでに整列済み)には O(n) で済みますが、平均および最悪のケースでは O(n²) である.

\subsection{空間計算量}

空間計算量は O(1) で、インプレースな操作が可能となる.
\end{document}
