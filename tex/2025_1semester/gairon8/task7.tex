\documentclass[uplatex]{jsarticle}
\usepackage{amsmath}
\usepackage[dvipdfmx]{graphicx}

\setcounter{tocdepth}{3}
\usepackage{float}
\usepackage{moreverb}
\usepackage{lscape}
%\pagestyle{empty}
%\usepackage{wrapfig}
%\usepackage{url}
%\usepackage{EasyLayout}

\usepackage{ascmac}
%\usepackage{fancybx}

%\pagestyle{myheadings}


\begin{document}

\title{情報工学概論第8回課題発展課題7}
\author{25G1051近藤巧望}
%\date{2025年5月22日}
\maketitle

\section{時間計算量}

アルゴリズムが処理を終えるまでにかかる時間を入力の大きさnに対してどのくらい増えるかで表したもの。
アルゴリズムの効率を評価するために使い,「ビッグオー記法(Big-O notation)」で表される.

\section{空間計算量}

アルゴリズムが実行中に使用するメモリ量を、入力サイズ n に対してどれくらい使うかで表したもの。「ビッグオー記法(Big-O notation)」で表される.

\end{document}