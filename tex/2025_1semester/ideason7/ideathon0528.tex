\documentclass[uplatex]{jsarticle}
\usepackage{amsmath}
\usepackage[dvipdfmx]{graphicx}

\setcounter{tocdepth}{3}
\usepackage{float}
\usepackage{moreverb}
\usepackage{lscape}
%\pagestyle{empty}
%\usepackage{wrapfig}
%\usepackage{url}
%\usepackage{EasyLayout}

\usepackage{ascmac}
%\usepackage{fancybx}

%\pagestyle{myheadings}



\begin{document}

\begin{flushright}
    \number\year 年\number\month 月\number\day 日
\end{flushright}

\begin{center}
    {\LARGE アイディアソンポスターセッション報告書}
\end{center}

\begin{flushright}
    グループ番号:21\\
    学生番号:25G1051\\
    氏名:近藤巧望
\end{flushright}


\section{背景・問題点・解決案}
% 2〜3行程度の文章で説明する.もう少し長くても可とする.
小学生の登下校の際は,大人の目から離れた場所での事件や事故が起きやすく大変危険である.
事件の発生については,地域での連携で子供たちを見守っているという場所もあるが,そのような対策だけでは事件や事故の発生を十分には防げない.
そこで,センサーで危険な車や人を検知して事故や事件の発生を未然に防ぐ「登下校ロボット」を提案した.



\section{質問と回答}
% 質問された内容と,それに対する理想的な回答や対応を箇条書きで記述する.似たような内容に対してまとめて回答を書いても良い.
% 以下は例なので,提出前に消すこと.
% 質問毎に1行空けて,文章上で段落を分けること.
・ポスターの問題点部分について,低学年児童の危機管理能力の低さなどが原因で起こる事故・事件のデータがないことを指摘された.
\\→歩行者側が原因で起こる事故のデータを示す必要があったため,次回はしっかりと記述する.そのため,事故に関しては信憑性があることを回答する.しかし,事件については効果的なデータを見つける事ができなかったため,練り直す必要がある.

・ロボットをタイヤ型にしないのか.
\\→ロボットの形において,明確な形を決めていたわけではなかったのに加え,脚で動くよりもタイヤで動く方が機体の耐久面や燃費も良くなるため,タイヤ型での導入の方向で答える.

・重さの制限はどうするのか.
\\→主に小学4年生あたりまでに向けて考えているため,4年生男子の平均体重35.7kgを元に45kgを限度とすると答える.

・扉のロックが自動か手動か.また,ロックは外からも開けられるのか.
\\→扉のロックは手動であると答える.ロックは,外から開けられると不審者から身を守れなくなるため,内側からロックを解除できる仕組みにすると答える.

・ロボットの置き場所をどうするのか.
\\→費用はかかってしまうが校内に置き場を設ける.

・電源が途中で切れてしまう場合どうするのか.
\\→自宅から学校までというように区間が定まっているため,往復の消費電力を計算して,一定の充電量を下回ると発信できないシステムを導入する.

・バスではダメなのか.
\\→スクールバスでは,時間が固定化されてしまい,交通の便が返って悪くなってしまう可能性がある.


\section{他のグループで参考になった点}
% 問題点に対する解決案が良かった,ポスターの作り方が良かった,説明が丁寧だったなど,今後の参考になるグループが有ったらグループ番号とその内容を書く.
% 以下は例なので,提出前に消すこと.
% 項目ごとに1行空けて,文章上で段落を分けること.
グループ1:解決策の部分が他の班よりも具体的に示されていて,再現性がとても高く感じられた.

グループ4:問題点と解決策がリンクしていた.

\end{document}



























