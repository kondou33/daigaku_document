\documentclass[uplatex,dvipdfmx]{jsarticle}

\usepackage{otf}
\usepackage[noalphabet]{pxchfon} 
\usepackage{stix2} 
\usepackage[fleqn,tbtags]{mathtools} 
\usepackage{amsmath}
\usepackage{url}
\usepackage{float} 

\setcounter{tocdepth}{3}
\usepackage{moreverb}
\usepackage{lscape}
\usepackage{ascmac}
\usepackage{xurl}
\usepackage{graphicx} % 画像挿入用

\begin{document}
\title{機能証明書}
\author{25G1051 近藤巧望}
\date{\today}
\maketitle
\tableofcontents
\newpage

\section{概要}
本機能証明書は,int型の配列を受け取ってその総和を返す関数と,
この関数を用いて配列の平均値を計算する関数の機能を証明するものである。
関数一覧を以下に示す。
\begin{itemize}
    \item sumArray: int型の配列とその要素数を受け取り,配列の総和を返す関数
    \item averageArray: int型の配列とその要素数を受け取り,配列の平均値を返す関数
\end{itemize}
\section{使用したコードとその出力例}
以下に,sumArray関数とaverageArray関数の使用例を示す。
\begin{verbatim}
int arr[] = {1,2,3,4,5};
int size = sizeof(arr) / sizeof(arr[0]);  // 5
int sum = sumArray(arr, size);
double average = averageArray(arr, size);
printf("Sum: %d\n", sum);              
printf("Average: %.2f\n", average);     
\end{verbatim}

上記のコードを実行した場合の出力例を以下に示す。
\begin{verbatim}
Sum: 15
Average: 3.00
\end{verbatim}

\section{おわりに}
本機能証明書では,sumArray関数とaverageArray関数の使用例を示し,その出力結果を確認することで,
両関数が正しく機能していることを証明した。
\end{document}